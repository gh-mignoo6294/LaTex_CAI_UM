\section{Riesgos}

\section{Principio de funcionamiento}
\label{sec:principio_funcionamiento_paradas_emergencia}
La función principal de una parada de emergencia es detener de manera inmediata y segura el funcionamiento de una máquina o sistema en caso de una situación peligrosa o de emergencia. Estas paradas están diseñadas para ser fácilmente accesibles y operables por cualquier persona que se encuentre cerca de la máquina, permitiendo una respuesta rápida para prevenir accidentes o daños.
El CAI cuenta con múltiples paradas de emergencia distribuidas estratégicamente alrededor de la máquina, así como una cuerda de vida que recorre las estaciones de pateadores en el perímetro exterior del recinto. Al activar cualquiera de estas paradas, se detienen de manera inmediata todos los motores de las cintas y se libera por completo las presiones los sistemas neumáticos, deteniendo todas las operaciones en curso y asegurando la seguridad del personal y del equipo. 
\begin{itemize}
    \item \textbf{PARADA CON LIBERACIÓN POR GIRO:} Se activan por medio del pulsado o presión del botón. Estas paradas requieren que el operador gire la cabeza del botón para liberarlo y permitir que vuelva a su posición original.
    \begin{center}
        \includegraphics[width=4cm]{2_Graphics/800FM-MT44_1000x1000.jpg}
    \end{center}

    \item \textbf{CUERDA DE VIDA:} Es un sistema de parada de emergencia que consiste en una cuerda o cable que recorre una zona específica, en este caso, las estaciones de pateadores del CAI. Al tirar de la cuerda en cualquier punto a lo largo de su recorrido, se activa la parada de emergencia, deteniendo inmediatamente todas las operaciones de la máquina. Para liberar la cuerda de vida se debe pulsar el botón de reset(botón azul) ubicado en el módulo de control de la propia cuerda de vida.
    \begin{center}
         \includegraphics[width=6cm]{2_Graphics/440E-LL5SE5--ISO.jpg}
    \end{center}
   
\end{itemize}
\section{¿Qué hacer luego de una parada de emergencia?}
\label{sec:que_hacer_luego_parada_emergencia}
Después de activar una parada de emergencia, es crucial seguir una serie de pasos para garantizar la seguridad y el correcto reinicio de la máquina:
\begin{enumerate}
    \item \textbf{Evaluar la situación}: Antes de proceder, asegúrese de que la situación que llevó a la activación de la parada de emergencia esté bajo control y que no haya riesgos inmediatos para el personal o el equipo.
    \item \textbf{Notificar al personal}: Informe a los supervisores y al personal relevante sobre la activación de la parada de emergencia y las circunstancias que la provocaron.
    \item \textbf{Inspeccionar la máquina}: Realice una inspección visual de la máquina para identificar cualquier daño o problema que pueda haber causado la necesidad de una parada de emergencia.
    \item \textbf{Resolver el problema}: Aborde cualquier problema identificado durante la inspección antes de intentar reiniciar la máquina. Esto puede incluir reparaciones, ajustes o limpieza.
    \item \textbf{Liberar la parada de emergencia}: Gire el botón de la parada de emergencia para liberarlo y permitir que vuelva a su posición original. En el caso de la cuerda de vida, presione el botón de reset en el módulo de control.
    \item \textbf{Reiniciar la máquina}: El reinicio del sistema de seguridad se ejecuta con los pulsadores azules que se encunetran en el panel HMI y en la puerta del tablero eléctrico principal. 
    \item \textbf{Retome las operaciones con normalidad}: Una vez reiniciado el sistema de seguridad, puede reanudar las operaciones normales de la máquina.
\end{enumerate}
\newpage
\section{Ubicación de las Paradas de Emergencia}
\label{sec:ubicacion_paradas_emergencia}
Las paradas de emergencia están estratégicamente ubicadas alrededor de la máquina para garantizar un acceso rápido en caso de una situación peligrosa. Al activar una parada de emergencia, se cortará inmediatamente la alimentación eléctrica de la máquina, deteniendo todas las operaciones en curso.
\begin{figure}[H]
    \centering
    \begin{tikzpicture}
        \node[anchor=south west, inner sep=0] (imagen) at (0,0) {
        \includegraphics[width=16cm]{2_Graphics/CAI_Vista0.PNG}};
        \node [SafetyRef] at (6.5,4) {1};
        \node [SafetyRef] at (14.9,1.3) {2};
        \node [SafetyRef] at (13.5,8.5) {3};
        \node [SafetyRef] at (3.5,8.5) {4};
        \node [SafetyRef] at (1.8,14.5) {5};
        \node [SafetyRef] at (14.5,11) {6};
    \end{tikzpicture}
    \caption{Ubicación de las paradas de emergencia en la máquina.}
    \label{fig:ubicacion_paradas_emergencia}
\end{figure}
\begin{table}[H]
    \renewcommand{\arraystretch}{1.5} % Da aire entre filas para que se lea mejor
    \begin{tabularx}{\textwidth}{|c|l|X|}\hline
    \rowcolor{gray!15} \textbf{Ref.} & \textbf{Componente} & \textbf{Ubicación} \\ \hline
    \safetyref{1} & Parada de Emergencia & En el frente del panel HMI\@. \\ \hline
    \safetyref{2} & Parada de Emergencia & En la puerta del tablero eléctrico principal. \\ \hline
    \safetyref{3} & Parada de Emergencia & En el inicio de la Cinta de Pateadores (C7). \\ \hline
    \safetyref{4} & Parada de Emergencia & En el final de la Cinta de Pateadores (C7). \\ \hline
    \safetyref{5} & Parada de Emergencia & En el final de la Rampa de Descartes (C8). \\ \hline
    \safetyref{6} & Cuerda de vida & Sobre las Estaciones de Pateadores 1 a 5. \\ \hline
\end{tabularx}
\caption{Leyenda de la figura \ref{fig:ubicacion_paradas_emergencia}.}
\label{tab:referencias_seguridad}
\end{table}

    \iffalse

        
        
    \fi
