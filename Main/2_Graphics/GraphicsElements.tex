\tikzset{
    startstop/.style={
        rectangle, rounded corners,
        minimum width=3cm, minimum height=1cm,
        text centered, draw=black, fill=red!30
    },
    io/.style={
        trapezium, trapezium left angle=70, trapezium right angle=110,
        minimum width=3cm, minimum height=1cm,
        text centered, draw=black, fill=blue!30
    },
    process/.style={
        rectangle, minimum width=3cm, minimum height=1cm,
        text centered, draw=black, fill=orange!30
    },
    decision/.style={
        diamond, minimum width=3cm, minimum height=1cm,
        text centered, draw=black, fill=green!30, aspect=2
    },
    arrow/.style={thick, ->, >=stealth
    },
    SafetyRef/.style={
        rectangle,              % Forma cuadrada
        rounded corners=2pt,
        draw=black,          % Borde negro
        fill=red,               % Relleno rojo
        very thick,          % Borde un poco más grueso para que resalte
        minimum size=6mm,    % Tamaño mínimo para que todos sean iguales
        font=\bfseries\small, % Fuente clara y en negrita
        text=white           % Color del número
    }
}
%Colors Setup
\definecolor{straight_belt_color}{HTML}{22B14C} % Verde bosque para las cintas rectas#
\definecolor{curve_belt_color}{HTML}{808080} % Gris para las cintas curvas

%Custom Commands
% Comando para automatizar los círculos de seguridad en tablas
\newcommand{\safetyref}[1]{%
    \begin{tikzpicture}[baseline=(char.base)]
        \node[SafetyRef, minimum size=6mm, font=\small\bfseries] (char) {#1};
    \end{tikzpicture}%
}
% Definición corregida para centrado vertical automático
\newcommand{\isosign}[2][0.7cm]{%
    \raisebox{-.3\height}{\includegraphics[height=#1]{2_Graphics/#2}}%
}