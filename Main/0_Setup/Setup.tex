% LTeX: enabled=false
%---------- LENGUAJE ----------%
\usepackage[spanish]{babel}

% Cambiar "Cuadro" por "Tabla"
\addto\captionsspanish{%
  \renewcommand{\tablename}{Tabla}%
  \renewcommand{\listtablename}{Índice de tablas}%
}
\raggedbottom
%---------- DEFINICIÓN DE FUENTES ----------%
\usepackage{fontspec}
\setmainfont{Sora}
[
    Path = ./0_Setup/fonts/,          % Ruta a la carpeta donde están los .ttf
    Extension = .ttf,         % Especificamos que son archivos TTF
    UprightFont = *-Regular,  % Sora-Regular.ttf
    BoldFont = *-Bold,        % Sora-Bold.ttf
    ItalicFont = *-ExtraLight, % Si no tenés itálica, podés usar un peso distinto
    BoldItalicFont = *-ExtraBold
]

%---------- CONFIGURACIÓN DE PÁGINA ----------%
\usepackage{geometry}
\geometry{
    a4paper,
    left=20mm, right=20mm,
    top=16mm, bottom=16mm,
    headheight=12mm,
    headsep=6mm,
    %footskip=20mm
}
\usepackage{float} % Permite el uso del modificador [H]
%---------- ENCABEZADOS Y PIE DE PÁGINA ----------%
\usepackage{fancyhdr}
\usepackage{fancyhdr}
\pagestyle{fancy}
\fancyhf{}
% ENCABEZADO
\fancyhead[LE,RO]{\footnotesize\bfseries\leftmark}
% PIE DE PÁGINA
\fancyfootoffset[L,R]{10mm}
\fancyfoot[LE,RO]{\bfseries \thepage} 
\fancyfoot[C]{\small \textit{CAI - Clasificador Automático de Ingresos}}
\renewcommand{\headrulewidth}{0.4pt}
\fancypagestyle{plain}{
    \fancyhf{}
    % ENCABEZADO
    \fancyhead[LE,RO]{\footnotesize\bfseries\leftmark}
    % PIE DE PÁGINA
    \fancyfootoffset[L,R]{10mm}
    \fancyfoot[LE,RO]{\bfseries \thepage} 
    \fancyfoot[C]{\small \textit{CAI - Clasificador Automático de Ingresos}}
    \renewcommand{\headrulewidth}{0.4pt}
}

\iffalse DefaultFontSizes
Comando,Tamaño (puntos),Uso común en tu manual
\tiny,6pt,Notas legales muy pequeñas.
\scriptsize,8pt,Subíndices o superíndices.
\footnotesize,10pt,Notas al pie de página.
\small,10.95pt,Encabezados y pies de página.
\normalsize,12pt,Cuerpo del texto general.
\large,14.4pt,Subsecciones.
\Large,17.28pt,Títulos de secciones.
\huge,20.74pt,"Etiqueta ""Capítulo X""."
\Huge,24.88pt,Nombre del Capítulo.
\fi

%---------- TABLAS ----------%
\usepackage{tabularx}    % Tablas con ancho fijo y columnas flexibles
\usepackage{booktabs}    % Líneas de tabla de alta calidad (superior, media, inferior)
\usepackage{colortbl}    % Para dar color a las celdas o filas si lo deseás

%---------- IMÁGENES Y GRÁFICOS ----------%
\usepackage{graphicx}
\usepackage{xcolor}
\usepackage{tikz}
\usetikzlibrary{shapes.geometric, arrows.meta, positioning, calc}

%---------- FORMATOS DE TÍTULOS ----------%
\usepackage{titlesec}
%\titlespacing{command}{left}{before-sep}{after-sep}
% 1. DEFINICIÓN DEL FORMATO (Obligatorio para que funcione el spacing)
\titleformat{\chapter}[display]
  {\normalfont\huge}           % 3. GLOBAL: Quitamos \bfseries de acá
  {\itshape\chaptertitlename\ \thechapter} % 4. ETIQUETA: Agregamos \itshape (Itálica)
  {10pt}                       % 5. SEPARACIÓN vertical
  {\bfseries\huge}             % 6. TÍTULO: Aquí sí ponemos negrita y más grande

% 2. AJUSTE DEL ESPACIADO
\titlespacing*{\chapter}
  {0pt}      % Margen izquierdo
  {-20pt}      % Arriba (Sube el título. Ajustá este valor si queda muy pegado al borde)
  {10pt}     % Abajo (Espacio antes de empezar el texto)

% 3. AJUSTE PARA SECCIONES (Más simple)
\titlespacing*{\section}
  {0pt}    % Izquierda
  {20pt}   % Arriba
  {10pt}   % Abajo

%----------- OTROS PAQUETES ----------%
\usepackage{pst-hazard}



%----------- DESHABILITADOS ----------%
\iffalse
\usepackage{changepage}
\titleformat{\chapter}[display]
  {\normalfont\huge\bfseries}{\chaptertitlename\ \thechapter}{20pt}{\Huge}
  

\titlespacing*{\section}{0pt}{0pt}{10pt}
\fi